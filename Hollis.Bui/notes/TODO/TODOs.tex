\documentclass[letter,10pt]{article}
\usepackage[utf8]{inputenc}
\usepackage[margin=1.0in]{geometry}
\usepackage{setspace}
\usepackage{listings}
\usepackage{color}
\usepackage[noend]{algpseudocode}
\usepackage{csquotes}
\usepackage{hyperref}
\def\UrlBreaks{\do\/\do-}
\hypersetup{
    colorlinks=true,
    linkcolor=blue,
    filecolor=magenta,      
    urlcolor=cyan,
    breaklinks=true,
}

\definecolor{dkgreen}{rgb}{0,0.6,0}
\definecolor{gray}{rgb}{0.5,0.5,0.5}
\definecolor{mauve}{rgb}{0.58,0,0.82}

\lstset{frame=tb,
  language=C++,
  aboveskip=3mm,
  belowskip=3mm,
  showstringspaces=false,
  columns=flexible,
  basicstyle={\small\ttfamily},
  numbers=none,
  numberstyle=\tiny\color{gray},
  keywordstyle=\color{blue},
  commentstyle=\color{dkgreen},
  stringstyle=\color{mauve},
  breaklines=true,
  breakatwhitespace=true,
  tabsize=3
}

%%%%%%%%%%%%%%% BEGIN LOCAL COMMANDS %%%%%%%%%%%%%%%%%%%
\newcommand{\sep}{\discretionary{}{}{}} % Used to help with text separation, hboxes.
%%%%%%%%%%%%%%% END LOCAL COMMANDS %%%%%%%%%%%%%%%%%%%

%opening
\title{Various Compiled TODOs and Notes}


\begin{document}

\maketitle
\newpage
\tableofcontents
\newpage

\section{Resources}

\subsection{Gilchrist Lab Materials}
\begin{itemize}
    \item Style guide: \url{https://github.com/clandere/RibModelDev/blob/master/desc/ribModel_specs.pdf}
    \item RFP Model pdf: \url{https://github.com/clandere/RibModelDev/blob/master/desc/rfp.model.pdf}
    \item Ensure these are the up-to-date PDFs: run \enquote{pdflatex} on the .tex files in the same repo.
    \item Gilchrist and Wagner paper (2006):
    \begin{itemize}
        \item \url{http://www.sciencedirect.com/science/article/pii/S0022519305003395}
    \end{itemize}
    \item Gilchrist, Chen, Shah, Landerer, Zaretzki paper (2015):
    \begin{itemize}
        \item \url{https://academic.oup.com/gbe/article-lookup/doi/10.1093/gbe/evv087}
        \item \url{http://biorxiv.org/content/early/2015/04/17/009670}
    \end{itemize}
\end{itemize}

\subsection{External Sources}
\begin{itemize}
    \item General info about Yeast gene nomenclature:
    \begin{itemize}
        \item \url{http://www.yeastgenome.org/help/community/nomenclature-conventions}
    \end{itemize}
    \item Weinberg Data paper link: \url{http://biorxiv.org/content/early/2015/07/06/021501}.
    \item Lareau Data paper link: \url{https://elifesciences.org/content/3/e01257}.
    \item Pop Data paper link: \url{http://www.ncbi.nlm.nih.gov/pmc/articles/PMC4300493/}.
\end{itemize}

\section{Short-Term Requirements}

\section{Maintenance-based Requirements (do as needed)}

\subsection{Unit Testing}
\begin{itemize}
    \item In general, each new function added to the framework should have unit testing done on it (as most new functions are likely to be simple).
    \begin{itemize}
        \item In particular, new functions to Utility.h, SequenceSummary.cpp, Gene.cpp, or Genome.cpp will likely belong here.
    \end{itemize}
\end{itemize}

\subsection{Commenting}
\begin{itemize}
    \item There should be a documentation block above each function in C++. This still needs to be done for many functions in C++. Focus especially on the arguments, the outputs, and how each function connects to each other. Note if it is exposed to RCPP, and what function \enquote{wraps} each other as they are progressively called.
    \item Similarly, some details need to be filled out in the R code. Ensure that the document() command is run so that \enquote{roxygen2} can properly fill out the documents.
\end{itemize}

\section{Long-Term Suggestions}


\section{R Code}

\end{document}

