\documentclass[12pt,hyperref]{labbook}
\usepackage[utf8]{inputenc}
\usepackage{graphicx}
\usepackage[margin=1.0in]{geometry}
\usepackage{setspace}
\usepackage{listings}
\usepackage{color}
\usepackage{array}
\usepackage{hyperref}
\usepackage[]{algorithm}
\usepackage[noend]{algpseudocode}
\usepackage{csquotes}
\usepackage{xspace}
\usepackage[normalem]{ulem} % For strikeout text
\usepackage{pdfpages} % allows inclusion of PDF files

\newcolumntype{P}[1]{>{\centering\arraybackslash}p{#1}}

\definecolor{dkgreen}{rgb}{0,0.6,0}
\definecolor{gray}{rgb}{0.5,0.5,0.5}
\definecolor{mauve}{rgb}{0.58,0,0.82}

%\textwidth=16.5cm
%mikeg: June 18, 2016 - Why is this being set? It should be set by geometry package
% Resolved June 27, 2016 (Hollis): After attempting to comment out, realized this function
% was used as a bandage on an abundance of overfull hboxes. 
% June 28, 2016 (Hollis): Added in the custom \sep command to fix hboxes.

% For verbatim quotes
\lstnewenvironment{verbquote}[1][]
  {\lstset{columns=fullflexible,
           basicstyle=\ttfamily,
           xleftmargin=2em,
           xrightmargin=2em,
           breaklines,
           breakindent=0pt,
           #1}}% \begin{verbquote}[..]
  {}% \end{verbquote}

\lstset{frame=tb,
  language=C++,
  aboveskip=3mm,
  belowskip=3mm,
  showstringspaces=false,
  columns=flexible,
  basicstyle={\small\ttfamily},
  numbers=none,
  numberstyle=\tiny\color{gray},
  keywordstyle=\color{blue},
  commentstyle=\color{dkgreen},
  stringstyle=\color{mauve},
  breaklines=true,
  breakatwhitespace=true,
  tabsize=3
}

%%%%%%%%%%%%%%% BEGIN LOCAL COMMANDS %%%%%%%%%%%%%%%%%%%
\newcommand{\DeltaEta}{\ensuremath{\Delta\eta}\xspace}
\newcommand{\DeltaM}{\ensuremath{\Delta M}\xspace}
\newcommand{\sep}{\discretionary{}{}{}} % Used to help with text separation, hboxes.
\newcommand{\LLik}{\ell}

%%%%%%%%%%%%%%% END LOCAL COMMANDS %%%%%%%%%%%%%%%%%%%


%%%%%%%%%%%%%%% BEGIN LOCAL CUSTOMIZATIONS %%%%%%%%%%%%%%%%%%%
\usepackage{etoolbox}
\makeatletter
%suppress pagebreaks between days
\patchcmd{\addchap}{\if@openright\cleardoublepage\else\clearpage\fi}{\par}{}{}
\makeatother 

%%%%%%%%%%%% END LOCAL CUSTOMIZATIONS %%%%%%%%%%%%%%%%%

\begin{document}

In the past we've run into this problem when we exponentiate large
negative LLik values, such as when we need to do a summation of the
probability of the set of possible outcomes.

For example, imagine $p_i = \exp[\LLik_i]/(\sum_j \exp[\LLik_j])$. If
the $\LLik$ values are all $< -312$, exponentiating these terms will lead to
0s and the sum of these 0s will also be 0.

This is a numerical issue, not a fundamental problem with the model.
The best solution for this is to define

\begin{equation}
\LLik_{\max} = \max(\vec{\LLik})
\end{equation}

and then rescale $\LLik$ terms by subtracting off $\LLik_{\max}$. Let
$\LLik_i^\prime$ represent the rescaled terms, such that $\LLik_i^\prime
= \LLik_i - \LLik_{\max}$.  They key result is that one of these terms
$\LLik_i^\prime$ term will always evaluate to 1 when exponentiated and
the rest will always evaluate to $< 1$ when exponentiated.

Why does this work?  Well,  subtracting a constant term from all the
exponents is equvalent to dividing all of them by $\exp[\LLik_{\max}]$ and
if we apply this division to both the numerator and denominator terms of
$p_i$, its value will be unchanged (but evaluatable). That is,

$p_i = \exp[\LLik_i]/(\sum_j \exp[\LLik_j])
    = \exp[\LLik_i^\prime]/(\sum_j \exp[\LLik_j^\prime])
$

I hope this helps.

Dr. Gilchrist

\end{document}
