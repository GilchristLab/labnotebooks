\documentclass[12pt,hyperref]{labbook}
\usepackage[utf8]{inputenc}
\usepackage{graphicx}
\usepackage[margin=1.0in]{geometry}
\usepackage{setspace}
\usepackage{listings}
\usepackage{color}
\usepackage{array}
\usepackage{hyperref}
\usepackage[]{algorithm}
\usepackage[noend]{algpseudocode}
\usepackage{csquotes}
\usepackage{xspace}
\usepackage[normalem]{ulem} % For strikeout text
\usepackage{pdfpages} % allows inclusion of PDF files

\newcolumntype{P}[1]{>{\centering\arraybackslash}p{#1}}

\definecolor{dkgreen}{rgb}{0,0.6,0}
\definecolor{gray}{rgb}{0.5,0.5,0.5}
\definecolor{mauve}{rgb}{0.58,0,0.82}

%\textwidth=16.5cm
%mikeg: June 18, 2016 - Why is this being set? It should be set by geometry package
% Resolved June 27, 2016 (Hollis): After attempting to comment out, realized this function
% was used as a bandage on an abundance of overfull hboxes. 
% June 28, 2016 (Hollis): Added in the custom \sep command to fix hboxes.

% For verbatim quotes
\lstnewenvironment{verbquote}[1][]
  {\lstset{columns=fullflexible,
           basicstyle=\ttfamily,
           xleftmargin=2em,
           xrightmargin=2em,
           breaklines,
           breakindent=0pt,
           #1}}% \begin{verbquote}[..]
  {}% \end{verbquote}

\lstset{frame=tb,
  language=C++,
  aboveskip=3mm,
  belowskip=3mm,
  showstringspaces=false,
  columns=flexible,
  basicstyle={\small\ttfamily},
  numbers=none,
  numberstyle=\tiny\color{gray},
  keywordstyle=\color{blue},
  commentstyle=\color{dkgreen},
  stringstyle=\color{mauve},
  breaklines=true,
  breakatwhitespace=true,
  tabsize=3
}

%%%%%%%%%%%%%%% BEGIN LOCAL COMMANDS %%%%%%%%%%%%%%%%%%%
\newcommand{\DeltaEta}{\ensuremath{\Delta\eta}\xspace}
\newcommand{\DeltaM}{\ensuremath{\Delta M}\xspace}
\newcommand{\sep}{\discretionary{}{}{}} % Used to help with text separation, hboxes.

%%%%%%%%%%%%%%% END LOCAL COMMANDS %%%%%%%%%%%%%%%%%%%


%%%%%%%%%%%%%%% BEGIN LOCAL CUSTOMIZATIONS %%%%%%%%%%%%%%%%%%%
\usepackage{etoolbox}
\makeatletter
%suppress pagebreaks between days
\patchcmd{\addchap}{\if@openright\cleardoublepage\else\clearpage\fi}{\par}{}{}
\makeatother 

%%%%%%%%%%%% END LOCAL CUSTOMIZATIONS %%%%%%%%%%%%%%%%%


\title{Notes for Undergraduate Research Work}
\author{Denizhan Pak}

\begin{document}

\maketitle
\newpage
\tableofcontents
\newpage

\labday{General}

\experiment{Purpose}

\experiment{Terminology}

\begin{itemize} 
	\item Codon Usage Bias (CUB): The variation between codons which are synonomous (code for the same amino acid) in a genome.
	\item Monte Carlo Markov Chain (MCMC): A technique used to create sets of data that is pseudorandomly distributed based on a given distribution. Utilizes random walks on a markov chain to generate random data for a Monte Carlo method.
	\item Mutation Bias: The variation between codon sequences caused by genetic mutations.
	\item Nonsense Error: An error in protein synthesis, when a stop codon is found prematurely, and the resulting protein is not what was initially expected.
	\item Pausing Time Model: A biological model to acquire information about protein translation. A freeze frame in which translation is stopped and the ribosome remains still. Locations of ribosomes can be analyzed. Ribosomes will spend more time on parts of mRNA that is less efficient to code and based on probabillity we can calculate which sets of codons are more innefficient based on the frequency of ribosomes that are attached to them.
\end{itemize}

\experiment{The Code Base}

\subsubsection{Models:}
The following is a list of the different types of MCMC models in this lab for the purpose of producing data that is reflective of CUB in a given genome or set of genomes. The models can be used to calculate the effects of synonymous substitutions on protein translation costs, gene expression levels and the strength of selection on CUB.
\begin{itemize}
	\item ROC: The Ribosome Overhead Cost model, it is the basic model for achieving the goal stated above. (described in Gilchrist et al. 2015).
	\item RFP: The Ribosome Footprinting model, is based on the ROC model however it is concerned with the position of ribosome using a Pausing Time model.
	\item PANSE: The Pausing and Nonsense Error model, this model accounts for nonsense errors by accoutning for the probabliity a codon is not reached due to nonsense errors in its random sampling it is an extension of the RFP model.
	\item FONSE: 
\end{itemize}

\labday{May 16, 2017 Notes}
\experiment{Time Breakdown}
\begin{itemize}
	\item 9 - 12: Looking through RFP Model.
	\item 1 - 3: Writing general notes for notebook
	\item 3 - 5: Ran a sample of RFP MCMC.
	\item Total: 7 hours spent for learning.
\end{itemize}
\experiment{First Run}
The following is a sample log file run with the objects initialized as follows:
\begin{itemize}
	\item genome(file = "rfp.counts.by.codon.and.gene.GSE63789.wt.csv", fasta = FALSE)
	\item parameter(genome = genome, sphi = 1, num.mixtures = 1, gene.assignment = rep(1, length(genome)), model = “RFP”)
	\item mcmc (samples = 50, thinning = 10, adaptive.width=50)
	\item model(parameter = parameter, model = "RFP")
	\item runMCMC(mcmc = mcmc, genome = genome, model = model)
\end{itemize}
The genome data used is from \textit{Pop et al (2014)}. The file was in "RFP" format as opposed to "Fasta" format. There were 50 samples with a thinning of 10, and the sphi value was set to a generic expectaion of 1.
\newline \newline
The output of the experiment was recorded in Log05.16.2017.txt.

\experiment{TO DO:}
\begin{itemize}
	\item Interpret and review log data
	\item Test MCMC with larger sample size
	\item Once new RFP model is pushed use as basis to start a working PANSE model.
\end{itemize}

\labday{May 17, 2017 Notes}
\experiment{Time Breakdown}
\begin{itemize}
	\item 9 - 11: Write Down ROC functions with descriptions.
	\item 11 - 1: Compare implmentation with function descriptions
	\item 2 - 5: Wrote down RFP functions with descriptions
	\item Total: 7 hours spent learning.
\end{itemize}
\experiment{Pseudocode for PANSE Model}
Take the functions in the description for the PANSE model and develop pseudocode for their implementation. To do this use ROC and RFP implementations as example.
\begin{enumerate}
	\item Write out functions and legends for RFP and ROC models.
	\item Map association between mathematical functions and implementation in models
\end{enumerate}
The notes taken can be found in the compositional lab notebook, scans can be found in the documents directory.

\labday{May 18, 2017 Notes}
\experiment{Time Breakdown}
\begin{itemize}
	\item 9 - 12: Looking through new RFP Model and comparing implementation to written equations
	\item 1 - 4: Wrote down all functions for PANSE and turned to Pseudocode
	\item 4 -5: Copied functions from RFP to PANSE and tested and ran code
	\item Total: 7 hours in Lab. 4 Hours spent working, 3 hours spent learning
\end{itemize}
\experiment{Pseudocode for PANSE Model (Continued)}
\begin{enumerate}
	\item Write out functions and legend for PANSE model
	\item Using maps create functions to associate with PANSE model
\end{enumerate}

\experiment{Rewrite PANSE}
Wroteover original files for PANSE, with new PANSE files all of which are directly copied from new RFP and have their functions renamed. Both of these were compiled and tested and were seen to work.

\labday{May 19, 2017 Notes}
\experiment{Time Breakdown}
\begin{itemize}
	\item 9 - 1: Turned PANSE functions into log to ease implementation like in RFP
	\item 2 - 3: Wrote pseudocode for PANSE Implementation using log version
	\item 3 - 5: Looked and found a log implementation of upper incomplete gamma function
	\item Total: 7 hours in Lab all spent working
\end{itemize}
\experiment{Turn PANSE Functions to Log}
Converted first 4 functions of PANSE model to loglikelihood from likelihood as it increases efficency and decreases processing time. The PANSE model requires an incomplete upper gamma function for caluclating the probabillity of elongation at a point j. The c++ standard library does not provide a generic incomplete upper gamma function. I have found two methods to implement this function:
\begin{enumerate}
	\item Using continued fractions, a method found on the git account of user Heng Li at url https://github.com/lh3/samtools/blob/master/bcftools/kfunc.c. This method uses a modified version of Lentz's algorithm to compute continued fractions to approximate the upper gamma function. There is currently an error in this implementation as aone of the coefficients is undefined somewhere with in the main loop body.
	\item The second is found on the wikipedia page for incomplete gamma functions. This implementation uses a combination of an evaluated gamma distribution and the gamma function to calculate the incomplete gamma result. The implementation on the wikipedia page is done in Excel. I will need to revise the implementation to work in c++.
\end{enumerate}
\experiment{TODO:}
\begin{enumerate}
	\item Debug continued fractions method.
	\item Implement wikipedia methon in c++.
	\item Test a few random variables with new upper gamma function and compare to R to see if results are accurate.
	\item Write a main function to test upper gamma on the order of 1000 times, with seeded random variables. Time both implementations compare results.
	\item Implement more efficient implementation in PANSE model.
\end{enumerate}

\labday{May 22, 2017 Notes}
\experiment{Time Breakdown}
\begin{itemize}
	\item 10 - 12: Tested Sam tools implementation and finished converting functions to Logs
	\item 1 - 3: Searched for new upper incomplete gamma function implementation.
	\item 3 - 5: Learned about continued fractional representation for functions.
	\item Total: 6 hours in Lab. 4 Hours spent working, 2 hours spent learning
\end{itemize}
\experiment{Finish Log Conversion}
All equations presented under the PANSE simulation in the documentation for the RFP model have been converted to their Logarithmis equivalents.
\experiment{Turn Equations to Pseudocode}
Have begun doing this to all equations for PANSE Model. Issue currently being faced is the lack of an upper incomplete gamma (UIG) function in the C++ standard library. So far the following techniques have been considered as possibillities:
\begin{itemize}
	\item Use Gamma DIstibution CDF: The Cumulative distribution function of a gamma distribution is proportional to the UIG, so we can use this function to calculate UIG.
	\item Use Continued Fractions: This method is approximating the distribution using continued fractions. Code is partially written but requires debugging.
	\item Use Lower Incomplete Gamma Function: The lower incomplete gamma function is easier to approximate because it's limit does not approach infinity. We could use this function and the regular Gamma Function to compute a value for UIG.
\end{itemize}

\labday{May 23, 2017 Notes}
\experiment{Time Breakdown}
\begin{itemize}
	\item 10 - 1: Searched for lower incomplete gamma implementation
	\item 1 - 3: Implemented lower incomplete gamma and wrote code to convert to upper
	\item 3 - 5: Tested code.
	\item Total: 7 hours in Lab all spent working
\end{itemize}
\experiment{Test Upper Gamma Function}
\begin{enumerate}
	\item Decided to use a lower incomplete gamma implementation to calculate upper incomplete gamma implementation.
	\item Tested new Gamma function comparing output to python and Mathematica implementations. The tests showed the implementation as accurrate for positive numbers greater than 0.
	\item Randomly generated input for function and timed for maximum of 10,000 iterations with an average of 0.3 second.
\end{enumerate}

\labday{May 24, 2017 Notes}
\experiment{Time Breakdown}
\begin{itemize}
	\item 9 - 10: Test new gamma function with values acquired from previous MCMC's
	\item 10 - 1: Looked for new Gamma function implementation
	\item 2 - 5: Looked for Cumulative distribution function of gamma distribution for c++ implementation.
	\item Total: 7 hours in Lab. 4 hours spent working, 3 hours spent learning
\end{itemize}
\experiment{Use Upper Incomplete Gamma Function}
There is a problem. The upper incomplete gamma function is undefined for negative values however in the description file we see that the gamma function may indeed recieve a negative value if the shape parameter for the distribution is greater than 1. To address this problem I will need to implement a new version of the upper incomplete gamma function that is defined at a negative value. The most likely candidate seems to be using the wikipedia suggested upper incomplete gamma distribution.

\labday{May 25, 2017 Notes}
\experiment{Time Breakdown}
\begin{itemize}
	\item 10 - 12: Learned about accurracy measurement in approximations
	\item 12 - 2: Read through GSL and R documentation on Incomplete Gamma Functions
	\item Total: 4 hours spent in Lab for learning.
\end{itemize}
Read into the implementation of upper incomplete functions in R and GSL libraries. Neither were able to provide insight into their implementation.

\labday{May 26, 2017 Notes}
\experiment{Time Breakdown}
\begin{itemize}
	\item 11 - 12: Looked at readings from Dr. Gilchrist
	\item 12 - 2: Look at implementations of upper incomplete in GSL.
	\item 2 - 5: Implement Continued fractions version of Upper incomplete estimation.
	\item Total: 6 hours in lab, 5 hours working 1 hour learning
\end{itemize}
\experiment{Upper Incomplete Gamma}
I was having trouble finding a working implementation of the upper incomplete gamma function. After reading on the GNU library I was not able to find the source code. Using the contnued fractions approximation describe \textit{Abramowitz and Stegun} I was able to write my own recursive algorithm which having tested with minimal data seems to work for both positive and negative values.
\experiment{TODO:}
\begin{itemize}
	\item Benchmarks: Need to test upper incomplete with a wider range of inputs
	\item Timing: Must measure and average timing for function to improve runtime
	\item Implement: Implement ress of probability of elongation calculation in PANSE Model.
\end{itemize}

\labday{May 30, 2017 Notes}
\experiment{Time Breakdown}
\begin{itemize}
    \item 12 - 12:30: Discussion with Dr. Gilchrist
    \item 12:30 - 2: Set up GNU Scientific Library
    \item 2 - 5: Work on testing script discussed with Dr Gilchrist
    \item Total: 5 hours spent in lab, 2 hours learning 3 hours working
\end{itemize}
\experiment{Benchmarking}
The working upper incomplete gamma function uses a 1000 iteration recursion process this is innefficient. To address this issue Dr. Gilchrist suggested benchmark testing. I am doing this using the following method.
\begin{itemize}
    \item Find comparable and well developed implementation (GSL).
    \item Develop a testing application.
    \item Test based on accurracy against GNU implementation.
    \item Test based on timing comparison against GNU implementation.
    \item Determine flexibillity in iteration count while maintaing accurracy.
\end{itemize}
I have begun developing the testing application. Under Dr. Gilchrist sugestion I will test with initial sample sizes ranging frm 500 to 1000.

\labday{May 31, 2017 Notes}
\experiment{Time Breakdown}
\begin{itemize}
    \item 11 - 1: Developed a set of functions and program to do testing for accurracy
    \item 1 - 3: Tested and edited gamma function for accurracy
    \item Total: 4 Hours spent in Lab working.
\experiment{Benchmark Upper Incomplete Gamma}
As discussed earlier worked on benchmarking my implementation of the Gamma Function. From the list above I have completed steps 2 and 3. After editing the code changing the number f iterations and adding a special case calculation the continued fraction is seeming to be as accurrate as the GSL implmenetation. The log files for the accurrcacy tests can be found in the Log_files directory.
I have more benchmarking to do although preliminarily the implementation seems to be fast enough, I will edit the testing function that has been developed to be multithreaded to allow for timing comparisons between both implementations. Depending on the results of this testing I will work on optimizing my implementation for increased speed efficiency. 
For the accurracy testing I used random number generation with the value for the $a$ being evenly distributed between $1 \land -1$. This is based on observations in the pdf of RFP that the $a$ value will be $1 - \alpha$ and from known $\alpha$ value observations it is between $0 \land 2$ meaning the resulting $a$ will be between $1 \land -1$. The value for $x$ is more arbitrary. It is set to a randomly distributed double between $0 \land 11$ this is because depending on $a$ most values for $x > 10$ will result in a return of $0$. I will test a working MCMC to understand a better value for $x$. Additionall I have provided a further reding on accurracy of a continued fraction in the further reading section.
\labday{June 1, 2017 Notes}
\experiment{Time Breakdown}
\begin{itemize}
    \item 12:30 - 1: Tested accurracy
    \item 1 - 3: Turned gamma function from recursive to iterative
    \item 3 - 5: Tested gamma function accurracy
    \item Total: 4.5 hours spent working
\end{itemize}
\experiment{Benchmarking}
I continued the benchmarking. During testing I notice a complication upon further investigation I found that the tested gamma function only matches gsl standard with in 7 - 8 decimal places. After learning this I conducted more tests to improve accurracy by increasing depth of the recursive fraction. There were memory issues caused by the overstacking of recursive calls. To improve efficiency and solve this problem I decided to turn the recursive function into an iterative one. To
do so I first turned the recursion from head to tail then converted the tail recursion to iteration. Accurracy has improved because we can do more iterations with the lower cost. There still seems to be deviation from the GSL standard however when comparing to the a 22 decimal accurrate calulcator meant to approximate the gamma function it seems to preform comparatively better than GSL implementation. To fursther investigate this I will develop a Python script to collect and compare data using the online calculator.
\end{document}
\labday{June 2, 2017 Notes}
\experiment{Time Breakdown}
\begin{itemize}
    \item 11 - 1: Wrote testing code
    \item 1 - 3: Tested speed and optimization mechanisms
    \item 3 - 4: Implemented upper gamma code in PANSE Model
    \item Total: 5 Hours spent working
\end{itemize}
\experiment{Testing}
Speed tests revealed positive results threads running 10000 iterations of the function clocked in less than 1 second. This is good news considereing the MCMC should not require more than 5000 iterations per run. In addition I wrote scripts to analyze the log files of the function find where the accurracy was well enough to decrease the number of iterations for future optimization.
\labday{June 5, 2017 Notes}
\experiment{Time Breakdown}
\begin{itemize}
    \item 10 - 11: Read through RFP and ROC code
    \item 11 - 1: Implemented probability of elongation and generalized gamma functions into PANSE Model
    \item 2 - 4: Tested and compiled implemented functions
    \item Total: 5 hours in lab, 1 hour learning, 4 hours working
\end{itemize}
\experiment{Testing newly implemented functions}
Functions seem to work successfully, however there is redundancy in the implementations because they were implemented directly from the RFP write up. There will be many ways to streamline these implementations and condense smaller loops, additionally new data thesees need to implemented which can be filled dynamically to decrease the number of iterations required for a single run such as storing elongation probabilities for specific codon positions so that they do not to be recalulcated.
Once the rest of the equations have been implemented and tested I will begin this method of optimization
\labday{June 6, 2017 Notes}
\experiment{Time Breakdown}
\begin{itemize}
    \item 10 - 12: Reworking implementations for PANSE code
    \item 1 - 3: Running and Testing PA code
    \item Total: 5 hours spent working
\end{itemize}
\experiment{Reworking Implementation}
For my hand written pseudo code I had not accounted for the resulting complication of the implementation of the upper incomplete gamma function, there are many reduncancies in the pseudocode that need to be eliminated. To address this issue I reworked and tried to find shortcuts for the equations given
\experiment{PA Testing}
Once I hit a point where I did not know how to implement some of the functions in the reworked form I decided to comb thrugh the PA code to check for similarities and find patterns of implementation. While testing I found reading and functional errors in the PA code. I have read through the source code to figure out this problem to no avail. I will use R and Rcpp debugging tools to continue trying to fix this problem
\labday{June 7, 2017}
\experiment{Time Breakdown}
\begin{itemize}
    \item 10 - 12: Mapping implementation
    \item 2 - 5: Reading and checking PA code
    \item Total: 2 Hours spent working, 3 hours spent learning
\end{itemize}
\experiment{Mapping Implementation}
Wrote pseudo code for the new implementation of the equations.
\experiment{Reading PA}
Problem still persists with not working Genome class implementation. Looked to through PA to compare to PANSE implementation, and use of the model in R calls.
\labday{June 9, 2017}
\experiment{Time Breakdown}
\begin{itemize}
    \item 10 - 11: Reading PA code
    \item 11 - 3: Running tests and working with PA Model
    \item 3 - 5: Changing heirarchy in PANSE Model
    \item Total: 7 hours in lab 6 Hours spent working, 1 hours spent learning
\end{itemize}
\experiment{Reading PA Code}
Tried to deciphr the relationship and generalizations from PA to PANSE. Talked with Dr Gilchrist about understanding that relationship.
\experiment{Running Tests with PA Model}
PA model is finally working again. I wanted to develop a better intuition of the workings of the library and as such ran multiple runs for the PA Model.
\experiment{Reworking Heirarchy}
Based on understanding of implementation and talking to Dr. Gilchrist, Cedric and Hollis, it makes sense to make the PANSE model a subclass of the PA model with changes. This however will e temporary as this relationship will then be reveresed as the PA model is a generalization of the PANSE Model.
