\documentclass[a4paper,10pt]{article}
%\documentclass[a4paper,10pt]{scrartcl}

\usepackage[utf8]{inputenc}

\title{}
\author{}
\date{}

\pdfinfo{%
  /Title    ()
  /Author   ()
  /Creator  ()
  /Producer ()
  /Subject  ()
  /Keywords ()
}

\begin{document}
In the past we've run into this problem when we exponentiate large
negative LLik values, such as when we need to do a summation of the
probability of the set of possible outcomes.

For example, imagine $p_i = \exp[\LLik_i]/(\sum_j \exp[\LLik_j])$. If
the $\LLik$ values are all < -312, exponentiating these terms will lead to
0s and the sum of these 0s will also be 0.

This is a numerical issue, not a fundamental problem with the model.
The best solution for this is to define

        $\LLik_\max = \max(\vec{\LLik})$

and then rescale $\LLik$ terms by subtracting off $\LLik_\max$. Let
$\LLik_i^\prime$ represent the rescaled terms, such that $\LLik_i^\prime
= \LLik_i - \LLik_\max$.  They key result is that one of these terms
$\LLik_i&^\prime$ term will always evaluate to 1 when exponentiated and
the rest will always evaluate to $< 1$ when exponentiated.

Why does this work?  Well,  subtracting a constant term from all the
exponents is equvalent to dividing all of them by $\exp[\LLik_max]$ and
if we apply this division to both the numerator and denominator terms of
$p_i$, its value will be unchanged (but evaluatable). That is,

$p_i = \exp[\LLik_i]/(\sum_j \exp[\LLik_j])
     = \exp[\LLik_i^\prime]/(\sum_j \exp[\LLik_j^\prime])
$

\end{document}
