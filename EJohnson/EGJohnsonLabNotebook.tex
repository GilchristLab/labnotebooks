%%%%%%%%%%%%%%%%%%%%%%%%%%%%%%%%%%%%%%%%%
% Daily Laboratory Book
% LaTeX Template 
%
% This template has been downloaded from:
% http://www.latextemplates.com
%
% Original author:
% Frank Kuster (http://www.ctan.org/tex-archive/macros/latex/contrib/labbook/)
%
% Important note:
% This template requires the labbook.cls file to be in the same directory as the
% .tex file. The labbook.cls file provides the necessary structure to create the
% lab book.
%  %%mikeg: Since labbook is installed shouldn't need to do this
% The \lipsum[#] commands throughout this template generate dummy text
% to fill the template out. These commands should all be removed when 
% writing lab book content.
%
% HOW TO USE THIS TEMPLATE 
% Each day in the lab consists of three main things:
%
% 1. LABDAY: The first thing to put is the \labday{} command with a date in 
% curly brackets, this will make a new page and put the date in big letters 
% at the top.
%
% 2. EXPERIMENT: Next you need to specify what experiment(s) you are 
% working on with an \experiment{} command with the experiment shorthand 
% in the curly brackets. The experiment shorthand is defined in the 
% 'DEFINITION OF EXPERIMENTS' section below, this means you can 
% say \experiment{pcr} and the actual text written to the PDF will be what 
% you set the 'pcr' experiment to be. If the experiment is a one off, you can 
% just write it in the bracket without creating a shorthand. Note: if you don't 
% want to have an experiment, just leave this out and it won't be printed.
%
% 3. CONTENT: Following the experiment is the content, i.e. what progress 
% you made on the experiment that day.
%
%%%%%%%%%%%%%%%%%%%%%%%%%%%%%%%%%%%%%%%%%
\documentclass[letterpaper,index=totoc,hyperref,openany]{labbook} % 'openany' here removes the gap page between days, erase it to restore this gap; 'oneside' can also be added to remove the shift that odd pages have to the right for easier reading

%----------------------------------------------------------------------------------------
%	PACKAGES AND OTHER DOCUMENT CONFIGURATIONS
%----------------------------------------------------------------------------------------

\usepackage[ 
%  backref=page, %incompatible with biblatex
  pdfpagelabels=true,
  plainpages=false,
  colorlinks=true,
  bookmarks=true,
  pdfview=FitB]{hyperref} % Required for the hyperlinks within the PDF
  
\usepackage{booktabs} % Required for the top and bottom rules in the table
\usepackage{float} % Required for specifying the exact location of a figure or table
\usepackage{graphicx} % Required for including images

%%Customizations by mikeg
\usepackage{marginnote}%provides ability to put notes in the margin using \marginnote{} coommand
\usepackage{ifthen}
\usepackage{amsmath,amssymb}
\usepackage{xspace}
\usepackage{listings} %provides lstlisting environment for typesetting code
\usepackage[natbib=true,backref=page]{biblatex} %alternative to natbib

\bibliography{/home/mikeg/BiBTeX/bibliography.full}

\usepackage{etoolbox}
\makeatletter
%suppress pagebreaks between days
\patchcmd{\addchap}{\if@openright\cleardoublepage\else\clearpage\fi}{\par}{}{}
%\patchcmd{\scr@startchapter}{\if@openright\cleardoublepage\else\clearpage\fi}{}{}{}
%remove numbering of experiments
%\renewcommand*\theexperiment{}
%\renewcommand*\thesubexperiment{}
\makeatother 

\newcommand{\HRule}{\rule{\linewidth}{0.5mm}} % Command to make the lines in the title page
\setlength\parindent{0pt} % Removes all indentation from paragraphs


%----------------------------------------------------------------------------------------
%	TITLE PAGE
%----------------------------------------------------------------------------------------

\frontmatter % Use Roman numerals for page numbers
\title{
\begin{center}
\HRule \\[0.4cm]
{\Huge \bfseries Research Journal \\[0.4cm] % Degree
\HRule \\[1.5cm]}
\end{center}
}
\author{\LARGE Elizabeth G. Johnson \\ \Large ejohns60@utk.edu \\[2cm]} % Your name and email address
\date{Beginning Auguestr 2016} % Beginning date



%\newexperiment{<abbrev>}[<short form>]{<long form>}
%Here, <abbrev> is the abbreviation that can be given later to make LATEX
%use the <long form> and <short form>. The short form is for index, table of
%contents and running title, and giving it is optional. When using the abbre-
%viation, specify it without prepending a backslash, i.e. \experiment{abbrev}.
%Abbreviations may contain any char except the backslash, the tilde ( ̃), comma

\newexperiment{NSE}{NSE SEMPPR}
\newexperiment{ROC}{ROC SEMPPR}
\newexperiment{RPF-ROC}[RPF-ROC]{Ribosome Profile Footprints: Pausing Model}
\newexperiment{RPF-NSE}[RPF-NSE \& ROC]{Ribosome Profile Footprints: NSE \& Pausing Model}
\newexperiment{FONSE}{FONSE SEMPPR}
\newexperiment{DIMCMC}[DIMCMC]{Doubly Intractable MCMC}
\newexperiment{Knight}{Student Learning, Jennifer Knight}
\newexperiment{SELAC}{Main SELAC model}
\newexperiment{Lab}{Lab Meeting}
\newexperiment{LSAs}{LSAs}
\newexperiment{Cedric}{Cedric Landerer}
\newexperiment{Mehmet}{Mehmet Aydeniz}
\newexperiment{SMBE Satellite Meeting on Protein Evolution in Denver}{SMBEDen}
%\newexperiment{shorthand}{Description of the experiment}


%COMMANDS
%%% Sort using M-x 'sort-lines'
\newcommand{\GTR}{GTR+$\Gamma$\xspace}
\newcommand{\LogN}{\ensuremath{\text{LogN}}\xspace}
\newcommand{\Lik}{\ensuremath{\text{\textbf{Lik}}}\xspace}
\newcommand{\LLik}{\ensuremath{\mathcal{L}}\xspace}
\newcommand{\Ne}{\ensuremath{{N_e}}\xspace}
\newcommand{\Piihat}{\ensuremath{\hat{\pi}_i}\xspace}
\newcommand{\Pii}{\ensuremath{\pi_{i}}\xspace}
\newcommand{\Pijhat}{\ensuremath{\hat{\pi}_j}\xspace}
\newcommand{\Pij}{\ensuremath{\pi_{j}}\xspace}
\newcommand{\Pivechat}{\ensuremath{\hat{\Pivec}}\xspace}
\newcommand{\Pivec}{\ensuremath{\Vec{\pi}}\xspace}
%\newcommand{\Pr}{\ensuremath{\text{Pr}}\xspace}
\newcommand{\Qmatrixa}{\ensuremath{\Qmatrix_a}\xspace}
\newcommand{\Qmatrix}{\mathbf{Q}\xspace}
\newcommand{\ROC}{\ensuremath{\text{ROC}}\xspace}
\newcommand{\Var}{\operatorname{Var}}
\newcommand{\var}{\Var}
\newcommand{\Wi}{\ensuremath{{W_i}}\xspace}
\newcommand{\Wj}{\ensuremath{{W_j}}\xspace}
\newcommand{\acivec}{\ensuremath{a\left(\cveci\right)}\xspace}
\newcommand{\acvecg}{\ensuremath{a\left(\vec{c}_{i,g}\right)}\xspace}
\newcommand{\acvecj}{\ensuremath{a\left(\cvecj\right)}\xspace}
\newcommand{\acvec}{\ensuremath{a\left(\Vec{c}\right)}\xspace}
\newcommand{\aip}{\ensuremath{a_{i,p}}\xspace}
\newcommand{\aivecg}{\ensuremath{{\avec}_{i,g}}\xspace}
\newcommand{\aivec}{\aveci}
\newcommand{\ajp}{\ensuremath{a_{j,p}}\xspace}
\newcommand{\ajvecg}{\ensuremath{{\ajvec}_{,g}}\xspace}
\newcommand{\ajvec}{\ensuremath{\Vec{a}_{j}}\xspace}
\newcommand{\aj}{\ensuremath{a__j}\xspace}
\newcommand{\alphac}{\ensuremath{\alpha_c}\xspace}
\newcommand{\alphap}{\ensuremath{\alpha_p}\xspace}
\newcommand{\alphavec}{\ensuremath{\Vec{\alpha}}\xspace}
\newcommand{\alphav}{\ensuremath{\alpha_v}\xspace}
\newcommand{\aobsvecg}{\ensuremath{{\avec}_{\text{obs},g}}\xspace}
\newcommand{\aobsvec}{\ensuremath{\Vec{a}_{\text{obs}}}\xspace}
\newcommand{\aobs}{\ensuremath{a_{\text{obs}}}\xspace}
\newcommand{\aoptip}{\ensuremath{\aopt_{i,p}}\xspace}
\newcommand{\aoptpg}{\ensuremath{\aopt_{p,g}}\xspace}
\newcommand{\aoptp}{\ensuremath{\aopt_p}\xspace}
\newcommand{\aoptvecg}{\ensuremath{{{\aoptvec}_g}}\xspace}
\newcommand{\aoptvec}{\ensuremath{\Vec{a}^*}\xspace}
\newcommand{\aopt}{\ensuremath{{a^*}}\xspace}
\newcommand{\aveci}{\ensuremath{\Vec{a}_i}\xspace}
\newcommand{\avecj}{\ensuremath{\Vec{a}_j}\xspace}
\newcommand{\avec}{\ensuremath{\Vec{a}}\xspace}
\newcommand{\avecopt}{\aoptvec}
\newcommand{\celegans}{\emph{C.~elegans}\xspace}
\newcommand{\cveci}{\ensuremath{\cvec_i}\xspace}
\newcommand{\cvecj}{\ensuremath{\cvec_j}\xspace}
\newcommand{\cvec}{\ensuremath{\Vec{c}}\xspace}
\newcommand{\deltaT}{\ensuremath{\delta t}\xspace}
\newcommand{\ecoli}{\emph{E.~coli}\xspace}
\newcommand{\Lklu}{\emph{L.~kluyveri}\xspace}
\newcommand{\bsubtilis}{\emph{B.~subtilis}\xspace}
\newcommand{\fij}{\ensuremath{f_{i,j}}\xspace}
\newcommand{\gen}{\ensuremath{\text{gen}}\xspace}
\newcommand{\jmax}{\ensuremath{{j_{\max}}}\xspace}
\newcommand{\kmax}{\ensuremath{{k_{\max}}}\xspace}
\newcommand{\muij}{\ensuremath{\mu_{i,j}}\xspace}
\newcommand{\phiROC}{\ensuremath{\phi_\ROC}\xspace}
\newcommand{\phig}{\ensuremath{\phi_{g}}\xspace}
\newcommand{\pij}{\ensuremath{p_{i,j}}\xspace}
\newcommand{\qij}{\ensuremath{q_{i,j}}\xspace}
\newcommand{\qji}{\ensuremath{q_{i,j}}\xspace}
\newcommand{\rib}{\ensuremath{\text{rib}}\xspace}
\newcommand{\cell}{\ensuremath{\text{cell}}\xspace}
\newcommand{\sphi}{\ensuremath{s_\phi}\xspace}
\newcommand{\scer}{\emph{S.~cerevisiae}\xspace}
\newcommand{\setG}{\ensuremath{\mathbb{G}}\xspace}
\newcommand{\setP}{\ensuremath{\mathbb{P}}\xspace}
\newcommand{\setC}{\ensuremath{\mathbb{C}}\xspace}
\newcommand{\setF}{\ensuremath{\mathbb{F}}\xspace}
\renewcommand{\ng}{\ensuremath{{n_g}}\xspace}
\newcommand{\researcher}{\subsubsection}
%---------------------------------------------------------------------------------------

\graphicspath{{./Figures//}} % double slash indicates search recursively within folder
%%Figures will be organized in subfolders by year/month
\DeclareGraphicsExtensions{.pdf, .png, .jpg} %prevent the need for using file extensions

\begin{document}

\frontmatter

\maketitle

\tableofcontents

\mainmatter % Use Arabic numerals for page numbers

%----------------------------------------------------------------------------------------
%	LAB BOOK CONTENTS
%----------------------------------------------------------------------------------------

% Blank template to use for new days:

%\labday{Day, Date Month \Year}

%\experiment{}

%Text

%-----------------------------------------

%\experiment{}

%\begin{figure}[H] % Example of including images
%\begin{center}
%\includegraphics[width=0.5\linewidth]{example_figure}
%\end{center}
%\caption{Example figure.}
%\label{fig:example_figure}
%\end{figure}

%Text

%----------------------------------------------------------------------------------------


%=======================================================================================
\labday{August 29th}
\experiment{Reversion}
%=======================================================================================
\begin{itemize}
\item Fixed Immune Fitting so optimizer does not get lodged in an inappropriate local minimum.
\item ISSUE 1: optimizer will not fit PAT OX at position 449
    \begin{itemize}
    \item{error A: ERROR IN BINOMIALPMF FUNCTION}
    \item{address A: a list instead of an array was being input, added a check for an array in binomial function}
    \end{itemize}
\item ISSUE 2: optimizer does not fit PAT OX 449 well
\begin{itemize}
\item{Not sure if stuck in local minima again}
\end{itemize}
%Graphics
%--------------------------------------------------------------------------
%\includegraphics[width=\textwidth]{Figures/2016/07/IMG_152.JPG}
%-------------------------------------------------------------------------


\end{itemize}


\experiment{Co-linkage}
\begin{itemize}
\item{ on hold}
\end{itemize}

\experiment{Physio-Chem Model}
\begin{itemize}
\item{ on hold}
\end{itemize}

%=======================================================================================

%=======================================================================================
\labday{August 31st}
\experiment{Reversion}
%=======================================================================================
Goal 8.31: Set up a brute force grid search for the immune optimization
\begin{itemize}

\item{ISSUE 8.31: some initial conditions seem to be causing some type of overflow in the freq-immune eq}
	\begin{itemize}
        \item{updated scipy v16 to v18 via conda to make use of updated optimizer that can better handle constraints}
        \item{works now for our subset of biologically reasonable initial conditions}
        \item{created randomized grid of initial conditions and ran against optimizer getting a new overflow error in exponent term}
 	\end{itemize}
\end{itemize}

%%%%%%%%%%%%%%%%%%%%%%%%%%%%%%%%%%%%%%%%%%%%%%%%%%%%%%%%%%%%%%%%%%%%%%%%%%%%%%%%%%%%%%%%%

%%%%%%%%%%%%%%%%%%%%%%%%%%%%%%%%%%%%%%%%%%%%%%%%%%%%%%%%%%%%%%%%%%%%%%%%%%%%%%%%%%%%%%%%%

%=======================================================================================
\labday{September 1st }
\experiment{Reversion}
%=======================================================================================
Goal 9.1.A: trouble shoot bugs in the brute force grid search for the immune optimization: Completed

Goal 9.1.B: integrate brute force grid search in optimization procedure

\begin{itemize}

\item{BUG ISSUE 1. 9.1.A: fixed handling of runtime error}
	\begin{itemize}
        \item{ID Error:RuntimeError "Optimal parameters not found"
        \item{wrote error handling exception for when optimizer fails to converge}
 	 \end{itemize}

\item{BUG ISSUE 2. 9.1.A: fixed overflow error}
	\begin{itemize}
    	\item{OverflowError: math range error in time dependent term}
	    \item{OverflowError error addressed by creating a bounding for lambda/alpha}
    \end{itemize}
    
\item{9.1.B: choose how to select local minima to feed into secondary likelihood optimization : semi-complete}

\item{9.1.B: test procedure on 4 types of immune reversions: incomplete}

\end{itemize}


%=======================================================================================
\labday{September 2nd }
\experiment{Reversion}
%=======================================================================================


Goal 9.1.B: integrate brute force grid search in optimization procedure

\begin{itemize}

\item{9.1.B: choose how to select local minima to feed into secondary likelihood optimization}
\item{9.1.B: test procedure on 4 types of immune reversions}
\item{Issue 9.1.B: immune equation is giving back nonsense trajectories that do not make structural sense with the equation}
	\begin{itemize}
	    \item{probem localized to the actual equation function}
	    \item{exits for specific parameter sets e.g.lambda,alpha,c,fo=[0.5816252442288061,
	     -0.0014841350965484241,
 	    0.5325576902608549,
 	    0.5174224206616665] but not for others}
	\end{itemize}

\end{itemize}


%=======================================================================================
\labday{September 3rd }
\experiment{Reversion}
%=======================================================================================


Goal 9.1.B: integrate brute force grid search in optimization procedure (On hold)

\newline \newline
Goal 9.2.A: Address bug that is leading to structurally nonsensical trajectories (fixed)

\newline \newline
Goal 9.2.B: Address bug that is preventing the optimizer from converging on a solution with lower c's.

\begin{itemize}

\item{Issue 9.2.A: immune equation is giving back nonsense trajectories that do not make structural sense with the equation}

	\begin{itemize}
	    \item{problem localized to the actual equation function}
	    \item{exits for specific parameter sets e.g.lambda,alpha,c,fo=[0.11443872001421844,-0.0014841350965484241,0.08043899604128835,0.07167719040936335] but not for 	others}
         \item{check what trajectory should look like in mathematica -for example fo=0.58 should give us an initial starting condition of 0.58 not 0.95: Different}
         \item{Identified problem: in log adjustment term changed $log(A +\epsilon)+log(A+B+\epsilon) to log(1)-log(1+B/A)$}
	\end{itemize}


\item{Issue 9.2.B: optimizer cannot converge on lower c solutions because parameter combinations in this region are throwing an error?}

	\begin{itemize}
        \item{fo:1.00085915995  c:  0.042187761197  lamb:  0.0864608069165  alpha:  -0.00333163118931} 9maxvalueforpower not defined) hardcoded maxvalue and retried.
        still cannot converge on the hand-fit optimimum}
        \item{tried brute force grid search again with a fixed low c (c=0.01) as an initial condition. Still does not find low c optimimums}
        \item{start curve fit at hand-fit low c optimum test to see where goes and will always go back to high c}
	\end{itemize}

\end{itemize}





%=======================================================================================
\labday{September 5th }
\experiment{Reversion}
%=======================================================================================


Goal 9.1.B: integrate brute force grid search in optimization procedure: complete
\newline \newline
Goal 9.5.A: Test immune-model on a variety of fits and determine if all trajectories are outputting un-biological c estimates: Not all, just ones where the immune response
contracts slowly. 






\begin{itemize}
\item{Issue 9.5.A: Some trajectories will not tolerate likelihood adjustments and the optimization hangs. The Python optimization function minimize calls a  wrapper to a FORTRAN function . When it fails the call just hangs and will not exit. There is not a good way as far as I can tell to look at what is causing the hang up. I do notice that it occurs when I have initial conditions that are too near to 1. (fo and lambda). Will try a time-out  multiprocessing /or callback function.  

http://stackoverflow.com/questions/24825441/
stop-scipy-minimize-after-set-time.}
\end{itemize}



\labday{September 5th }
%=======================================================================================
\experiment{Reversion}
\researcher{}
\begin{description}
  \item[Project State:]
    \begin{itemize}
    
    \item{Vitaly is concerned about the fact that the initial condition parameter $fo$ is not fitting well. Wants me to trouble shoot}
    \item{High un-biological c-values are due to the fact that the immune response is contracting slowly and the cost differential has to be very high in order for 	the revertant to emerge in an environment where there is still an immune response targeted against it. I suggested creating an exhaustion term for the effector cells to address the fact that the immune response seems to be contracting and expanding at different rates during the 500 day time-course. This exhaustion term should hopefully give us more biological c-values. In this formulation,
	the $\kappa$ clearance term (which denotes how well the effectors clear the virus) is modulated to account for exhaustion as follows in equation below text.}
	
$$\text{Reversion with Effector Response and Exhaustion of the Effector Response}$$


$$\diamond\diamond\text{ODE Equations}\diamond\diamond$$

$$\dot W=  rW- \kappa(t) E W -\mu_w rW + \mu_m r(1-c)M $$
$$\dot M = r(1-c)M  +\mu_w rW -\mu_m r(1-c)M $$
$$\dot E =\alpha E$$




$$\kappa=
\begin{cases}
\hphantom{-}\kappa_0 e^{-\rho t} &\text{if $t \geq t_{exh}$},\\[2ex]
\hphantom{-}\kappa_0  &\text{if $ t< t_{exh}$}.
\end{cases}$$



$$\diamond \diamond \text{Closed form Solution}\diamond \diamond$$

$$f_w(t)=\dfrac{f_0 e^{\frac{\lambda}{\alpha}} } {(1-f_0) e^{\frac{\lambda e^{\alpha t}}{\alpha}-crt}+ f_0 e^{ \frac{\lambda}{\alpha}} }$$

$$\lambda=\kappa E_0$$

$$\kappa=
\begin{cases}
\hphantom{-}\kappa_0 e^{-\rho t} &\text{if $t \geq t_{exh}$},\\[2ex]
\hphantom{-}\kappa_0  &\text{if $t < t_{exh}$}.
\end{cases}$$


\begin{table}[H]
\centering
\caption{}
\label{my-label}
\begin{tabular}{|l|l|}
\hline
                 & Parameters and State Variables               \\ \hline
$$M$$            & rna copies of mutant variant per ml of blood \\ \hline
$$W$$            & rna copies of wild variant per ml blood      \\ \hline
$$E$$            & effector response of immune system           \\ \hline
$$f_w$$          & frequency of wild variant                    \\ \hline
$$f_0$$          & initial frequency of wild variant            \\ \hline
$$\mu_w$$        & mutation rate of wild into mutant population \\ \hline
$$\mu_m$$        & mutation rate of mutant into wild population \\ \hline
$$c$$            & replicative cost of mutation                 \\ \hline
$$r$$            & growth constant of wild variant              \\ \hline
$$\kappa$$       & killing rate of effector cells               \\ \hline
$$\alpha$$       & expansion rate of effector cells             \\ \hline
$$\rho$$         & exhaustion rate of effector cells            \\ \hline
$$t_{exh}$$      & beginning of exhaustion phase                \\ \hline
\end{tabular}
\end{table}
    
    \end{itemize}
  \item[Current Tasks:]{ Prototype new variation of model in mathematica and fit to 4 test cases. Decide if $fo$ value can be fit better.}
  \item[Future Tasks:] {Create a python version of model and fit to 4 test cases.}
  \end{description}




\labday{September 7th }
%=======================================================================================
\experiment{Reversion}
\researcher{}
\begin{description}
  \item[Project State:]
    \begin{itemize}
    
    \item{Vit wants me to write the equation with the actual delay in it, not restart our function with a new $\alpha$ and new initial conditions}
    
    \item{Vit thinks we are getting bad estimates of c because the data is bad for the examples I showed him. I think this might be outside my realm of control.}
    \end{itemize}
  \item[Current Tasks:]
  \item{have ODE with delay. Just need to finish solving it.}
  \item[Future Tasks:] {rerun fits with new equation that incorporates immune exhaustion}
  \end{description}
  
  
  $$\dot W=  rW- \kappa E W -\mu_w rW + \mu_m r(1-c)M $$
$$\dot M = r(1-c)M  +\mu_w rW -\mu_m r(1-c)M $$
$$\dot E =\alpha E$$




$$\dot \kappa=
\begin{cases}
 -\rho \kappa   &\text{if $t \leq t_{exh}$},\\[2ex]
\hphantom{-}  0  &\text{otherwise}.
\end{cases}$$
  
  $$\dot{z}=\Big( \dfrac{W}{M} \Big)\dfrac{d}{dt} =
\begin{cases}
\hphantom{-} \dfrac{M \Big( rW- (\kappa_0 e^{\rho  t_{exh}})  E_0 e^{(\alpha -\rho) t}  W \Big)}{M^2}-\dfrac{ W \Big(r(1-c)M \Big)}{M^2} &\text{if $t \geq t_{exh}$},\\[2ex]
\hphantom{-} \dfrac{M \Big( rW- \kappa E_0 e^{\alpha t} W \Big)}{M^2}-\dfrac{ W \Big(r(1-c)M \Big)}{M^2}  &\text{if $t < t_{exh}$}.
\end{cases}$$




\labday{September 13th }
%=======================================================================================
\experiment{Reversion}
\researcher{}
\begin{description}
  \item[Project State:]{
    \begin{itemize}
    \item{I need to add a callback function for the immune equation optimization and then the optimization should be able to run automatically without hanging over multiple difficult patient trajectories. In my opinion the model is terrible and we are just not going to get reasonable fits with the immune resposne term we have. For those patient positions that prove to be too difficult to get a reasonable fit and hang, I will simply flag and make a personal call on whether or not to reject the fit. I also need at some point figure out where the immune fits should be integrated into the hash table.} 
    \end{itemize}
  \item[Current Tasks:] 1.5 hrs I need to get the callback function working today. Success. I had to use an optimization algorithm other that COBYLA and swap it out for Nelder-Mead.
  
  \item[Future Tasks:]
  \begin{itemize}
  \item{I cleaned up my code so I need to make sure it still works up to the brute force optimization}
  \item{I need to run the new optimizer with the callback through its paces and  and make sure it does not have any problems}
  \item{I need to integrate my immune freq fits into the hash table and regenerate my fitness landscape}
  \end{itemize}
  \end{description}
  
  
  \experiment{Physio-Chemical Distance}
\researcher{}
\begin{description}
  \item[Project State:]{
    \begin{itemize}
    \item{ Currently, I have a fitness map generated from our physio-chemical model but it uses amino acid distance values taken directly from the grantham paper. What I would like to do instead, is derive those amino acid distance values for HIV and describe my distance function instead with three weighting parameters for differences between polarity, molecular weight and side-chain composition. Right now, I need to integrate my function for physio-chemical distance (written in terms of these properties) into the optimization so I can identify what my three weighting terms are. }
    \end{itemize}
    
  \item[Current Tasks:] integrate distance function into code
  \item[Future Tasks:]
  \begin{itemize}
  \item{streamline optimization so it is faster (pre-generate all values)}
  \item{create a a function that can easily generate toy subsets of sequences that will allow me to verify that my code works.}
   landscape}
  \item{I need create outer loop for testing different weighting parameters for the distances}
  \item{Need to verify my optimization schema works by testing it on the toy sequences}
  \item{Identify which optimization algorithm works best on toy sequences}
  \item{implement a basin-hopping algorithm to verify we have reached the real local optimum}
  \item{test optimization schema on real subset of the gag sequences}
  \item{run optimization schema on entire protein}
  \end{itemize}
  \end{description}

\printbibliography

\end{document}

\experiment{}
\researcher{}
\begin{description}
  \item[Project State:]
    \begin{itemize}
    \end{itemize}
  \item[Current Tasks:] 
  \item[Future Tasks:] 
  \end{description}
