\documentclass[12pt]{labbook}
\usepackage[utf8]{inputenc}
\usepackage{graphicx}
\usepackage{csquotes}
\usepackage{float}
\usepackage[margin=1.0in]{geometry}
\usepackage{amsmath}
\usepackage{setspace}
\usepackage{listings}
\usepackage{color}
\usepackage{array}
\usepackage{times}
\usepackage{amssymb, amsmath}
\usepackage{fancyhdr}
\usepackage[]{algorithm}
\usepackage[noend]{algpseudocode}
\usepackage{courier}

\newcolumntype{P}[1]{>{\centering\arraybackslash}p{#1}}

\definecolor{dkgreen}{rgb}{0,0.6,0}
\definecolor{gray}{rgb}{0.5,0.5,0.5}
\definecolor{mauve}{rgb}{0.58,0,0.82}

\textwidth=16.5cm

\lstset{frame=tb,
  language=Java,
  aboveskip=3mm,
  belowskip=3mm,
  showstringspaces=false,
  columns=flexible,
  basicstyle={\small\ttfamily},
  numbers=none,
  numberstyle=\tiny\color{gray},
  keywordstyle=\color{blue},
  commentstyle=\color{dkgreen},
  stringstyle=\color{mauve},
  breaklines=true,
  breakatwhitespace=true,
  tabsize=3
}

\title{Notes for Undergraduate Research Work}
\author{Alan Dixon}

\begin{document}

\section{June 6, 2016 Notes}

\begin{itemize}
    \item Discovered and fixed two bugs involving an uninitialized selection vector and Mphi values
    \item Said bug fixes resolved the issue of the program crashing around iteration 2500
    \item Despite the fixed bugs resolving that issue, log likelihood values still plummet to values eventually reaching 10e-23 which causes the not a number error, causing the MCMC algorithm to halt.
    \item Due to Newton failing to execute my job request, I had to run the R script on my laptop so I was only able to run it once so more testing is needed but for the meantime it appears it takes until around iteration 42000 for the program to crash.
    \end{itemize}

TODO:\\
    Work on psuedocode\\
    Remove redundancies\\
    Determine if Sphi prior is being used\\
    Improve varialbe names and documentation\\
    Document the R scripts\\
    
\section{June 7, 2016}

\begin{itemize}
    \item Removed exit statements from FONSEParameter.cpp, PANSEParameter.cpp, and RFPParameter.cpp in order to eliminate warnings when running the check in R
    \item Downloaded all of the necessary files to my Gauley account.
    \item Running FONSE with simulated data and b = 0.001 led to "nan" being reached much more quickly than on my laptop, around iteration 5500
    \item Said issue may be a cause of R not working correctly on Gauley as compared to my laptop. Multiple errors and missing packages were encountered. These errors made it impossible to check any plots.
    \item Two new Items for the TODO list: Continue troubleshooting R on Gauley and alter Jeremy's .sge scripts so that they no longer email him when I run a job on Newton
\end{itemize}

\section{June 8, 2016}

\begin{itemize}
    \item Changed the .sge scripts for Newton so that Jeremy no longer receives an email upon the completion of a job
    \item Changed a function in FONSEModel.cpp to match it's corresponding function in ROCModel.cpp. In particular, I changed a long series of divisions to on division and a series of multiplications because multiplication is a faster operation.
    \item Fixed a typo in the documentation of ROCModel.cpp
    \item Worked out some kink experienced while running R scripts on Gauley. In particular, finally made it so that the plots actually showed up and worked.
    \item Formulated the hypothesis that the reason we are experiencing a drop in log likelihood comes from a bug involving either deltaM, deltaOmega, or phi after observing the trace plots.
\end{itemize}

\section{June 13, 2016}

\begin{itemize}
    \item Fixed an error in FONSEModel.cpp involving a couple of missing semicolons
    \item Fixed an error in the .tex file for the notes where using the hyperref package kept the notes from compiling
    \item After running FONSE again, with simulated data and b = 0.001, it seems the errors in log likelihood are no longer deterministic. Values still plummet, and the traces suggest that both mutation and selection values may be the cause. Further testing and observation is required.
    \item Found a difference between FONSEModel and ROCModel where ROC would take mutation prior into account and FONSE wouldn't. This might part of the issues causing log likelihood values to plummet, but testing is required.
    \item Uploaded both the new .tex file and its corresponding pdf just in case there's another error that doesn't keep me from compiling on my machine but might keep another from compiling on a different machine.
\end{itemize}

\section{June 14, 2016}

\begin{itemize}
    \item The addition of the mutation prior into FONSE seems to have stopped the rapid descent of the Log Likelihood values. After running it once with 1000 samples and 1 thining, the log likelihood stayed somewhat stable after starting low and rising to about -20000. The vaules were also similar when running with 1000 samples and 10 thining. More extensive testing is required to ensure that the plummeting problem is finally resolved.
    \item Although the log likelihood values are starting to come together, there is an issue observed in both the mutation and selection traces where certain codons are not being accepted. 
    \item Conducted further comparisons on FONSE and ROC to remove redundancies and semantic oversight.
\end{itemize}
\end{document}













